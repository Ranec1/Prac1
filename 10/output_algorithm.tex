\hspace{1mm}\textcolor{blue}{function} [ E ] = bisection( A,a,b,tol ) \\ 
\hspace{1mm} \\ 
\hspace{1mm}b\_copy = b; \\ 
\hspace{1mm}n = getNBeigenvalues(A,b)-getNBeigenvalues(A,a); \\ 
\hspace{1mm}E = zeros(n,1); \textcolor{green}{\% The vector that contains all eigenvalues }\\ 
\hspace{1mm} \\ 
\hspace{1mm}\textcolor{blue}{while} (n$>$0) \\ 
\hspace{1mm}\indent \textcolor{green}{\% Find the smallest eigenvalue in the interval }\\ 
\hspace{1mm}\indent \textcolor{blue}{while} (b-a$>$tol) \\ 
\hspace{1mm}\indent \indent half = (b+a)/2; \\ 
\hspace{1mm}\indent \indent \textcolor{blue}{if} (getNBeigenvalues(A, half) - getNBeigenvalues(A, a) == 0)  \\ 
\hspace{1mm}\indent \indent \indent \textcolor{green}{\% Geen eigenwaarde in onderste helft dus bovenste nemen. }\\ 
\hspace{1mm}\indent \indent \indent a = half; \\ 
\hspace{1mm}\indent \indent \textcolor{blue}{else} \\ 
\hspace{1mm}\indent \indent \indent b = half;    \textcolor{green}{\% Default: verklein interval  }\\ 
\hspace{1mm}\indent \indent \textcolor{blue}{end} \\ 
\hspace{1mm}\indent \textcolor{blue}{end} \\ 
\hspace{1mm}\indent \textcolor{green}{\% eigenwaarde is gevonden }\\ 
\hspace{1mm}\indent E(n) = (a+b)/2; \\ 
\hspace{1mm}\indent n = n-1; \\ 
\hspace{1mm}\indent  \\ 
\hspace{1mm}\indent \textcolor{green}{\% opnieuw itereren met het interval zonder kleinste eigenwaarde van }\\ 
\hspace{1mm}\indent \textcolor{green}{\% vorige interval }\\ 
\hspace{1mm}\indent  \\ 
\hspace{1mm}\indent a = b; \\ 
\hspace{1mm}\indent b = b\_copy; \\ 
\hspace{1mm}\indent  \\ 
\hspace{1mm}\textcolor{blue}{end} \\ 
\hspace{1mm} \\ 
\hspace{1mm}\textcolor{blue}{end} \\ 
\hspace{1mm} \\ 
