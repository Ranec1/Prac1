\documentclass{article}
\usepackage[utf8]{inputenc}
\usepackage{mathtools}
\usepackage{amssymb}

\title{Practicum numerieke}
\author{Ranec Belpaire, Rik Bauwens }
\date{April 2017}

\begin{document}

\maketitle

\section{Opgave 9}
Gegeven een symmetrische matrix $A \in \mathbb{R}^{m \times m}$, met $A^{(1)}, \dots, A^{(m)}$ zijn boven-linkse vierkante deelmatrix van dimensies $1, \dots, n$. Het kan bewezen worden dat de eigenwaarden van deze matrices voldoen aan de $interlacing$ eigenschap. De eigenwaarden van matrix $A^{k}$ voldoen aan $\lambda^{(k)}_{1} < \lambda^{(k)}_{2} < \dots < \lambda^{(k)}_{k}$.  De $interlacing$ eigenschap stelt dat de ongelijkheden $$\lambda^{(k+1)}_{j} < \lambda^{(k)}_{j} < \lambda^{(k+1)}_{j+1}$$ gelden voor $k=1,2,\dots,m-1$ en $j=1,2,\dots,k-1$.

\end{document}

